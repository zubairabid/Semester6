\documentclass[a4 paper]{article}
% Set target color model to RGB
\usepackage[inner=2.0cm,outer=2.0cm,top=2.5cm,bottom=2.5cm]{geometry}
\usepackage{setspace}
\usepackage[rgb]{xcolor}
\usepackage{verbatim}
\usepackage{subcaption}
\usepackage{amsgen,amsmath,amstext,amsbsy,amsopn,tikz,amssymb,tkz-linknodes}
\usepackage{fancyhdr}
\usepackage[colorlinks=true, urlcolor=blue,  linkcolor=blue, citecolor=blue]{hyperref}
\usepackage[colorinlistoftodos]{todonotes}
\usepackage{rotating}
\usepackage{longtable}
\usepackage{booktabs}
\usepackage{multirow}
\input{macros.tex}


\begin{document}
\homework{Course Project Report}{Due: 27/04/20}{Dr. Parameshwari}{}{Zubair Abid}{20171076}

\textbf{Instructions}: Read all the instructions below carefully before you start working on the report, and before you make a submission.
\begin{itemize}
    \item Please typeset your submissions in \LaTeX. Use the template provided for your answers. Please include your name and Roll number with submission. 
    \item This report is due on 27 April at 11:55 PM.
    \item No extensions will be given for the submission under any circumstances.
    \item Submissions via any other method other than Moodle submission will be deemed invalid.
    \item Plagiarism of any sort will not be tolerated. Strict action will be taken for those caught in plagiarism.
\end{itemize}

\section{Tags}

Discussing the results of AnnCorra and UD tagging in my chosen Indian Language,
Bengali.

\subsection{UD}

Based on the results of tagging 100 sentences with UD tags, we present the
following results.\\
It is to be noted that a number of issues arise from tying to tag bengali with
UD tags, including but not limited to a dearth of tags for specific use cases,
requiring the use of a general tag covering all the various nuanced use cases
that are present in the language. A detailed analysis will be provided later.

\begin{longtable}{cp{0.35\textwidth}p{0.35\textwidth}}
    \toprule
    \textit{Tag} & \textit{Examples} & \textit{Linguistic Cues} \\ 
    \toprule
    % \multirow{2}{*}[-1em]{nsubj}
    \addlinespace[0.5em]
    \textbf{nsubj}
                 & \textbf{bijepi} korbe kal shukrobar
                 & POS: Noun\\
    \addlinespace[0.5em]
                 & ekti porjay porjonto \textbf{tader} shekhano hoy
                 & Suffixes: -er, -r (possessive), \\
    \addlinespace[0.5em]
                 & \textbf{tini} tader byaboharic ingreji ou gonit poraten
                 & POS: Pronoun\\
    \midrule
    \addlinespace[0.5em]
    \textbf{obj}
                 & tini \textbf{tader} byaboharic ingreji ou gonit shekhan
                 & POS: Pronoun; Suffixes: der\\
    \addlinespace[0.5em]
                 & tar biddaloyer nirdhishto kono \textbf{pathshuchi} ba boi nei
                 & POS: Noun \\
    \midrule
    \addlinespace[0.5em]
    \textbf{iobj}
                 & ekta \textbf{porjay} porjonto tader shekhano hoy
                 & Multiple Example 2\\
    \addlinespace[0.5em]
                 & Multiple Example 1 
                 & Multiple Example 2\\
    \midrule
    \addlinespace[0.5em]
    \textbf{csubj}
                 & Multiple Example 1 & Multiple Example 2\\
    \addlinespace[0.5em]
                 & Multiple Example 1 & Multiple Example 2\\
    \midrule
    \addlinespace[0.5em]
    \textbf{ccomp}
                 & Multiple Example 1 & Multiple Example 2\\
    \addlinespace[0.5em]
                 & Multiple Example 1 & Multiple Example 2\\
    \midrule
    \addlinespace[0.5em]
    \textbf{ccomp}
                 & Multiple Example 1 & Multiple Example 2\\
    \addlinespace[0.5em]
                 & Multiple Example 1 & Multiple Example 2\\
    \midrule
    \addlinespace[0.5em]
    \textbf{ccomp}
                 & Multiple Example 1 & Multiple Example 2\\
    \addlinespace[0.5em]
                 & Multiple Example 1 & Multiple Example 2\\
    \midrule
    \addlinespace[0.5em]
    \textbf{ccomp}
                 & Multiple Example 1 & Multiple Example 2\\
    \addlinespace[0.5em]
                 & Multiple Example 1 & Multiple Example 2\\
    \midrule
    \addlinespace[0.5em]
    \textbf{ccomp}
                 & Multiple Example 1 & Multiple Example 2\\
    \addlinespace[0.5em]
                 & Multiple Example 1 & Multiple Example 2\\
    \midrule
    
\caption{Discussion on UD tags}
\label{tab:udtable}
\end{longtable}

\subsection{AnnCorra}
Do the same for AnnCorra.

\begin{longtable}{cp{0.35\textwidth}p{0.35\textwidth}}
    \toprule
    \textit{Tag} & \textit{Examples} & \textit{Linguistic Cues} \\ 
    \toprule
    % \multirow{2}{*}[-1em]{nsubj}
    \addlinespace[0.5em]
    \textbf{k1}
                 & \textbf{bulabulite} dhAna kheYechhe
                 & POS: Noun; Suffixes: -te\\
    \addlinespace[0.5em]
                 & \textbf{AmAra} shIta karachhe
                 & POS: Pronoun, Suffixes: -Ara\\
    \addlinespace[0.5em]
                 & \textbf{AmAdera} siTi kaleja khuba bhAla
                 & POS: Pronoun; Suffixes: -r (possessive)\\
    \addlinespace[0.5em]
                 & AnandamaTha \textbf{baNkimachandra} kartRRika rachita
                 & POS: Noun; Postposition: kartRRika (also dbArA and diYe)\\
    \midrule
    \addlinespace[0.5em]
    \textbf{k2}
                 & bhUmikampa sAjAno gochhAno \textbf{shaharaTA} dhbansa karala
                 & POS: Noun; Suffixes: -TA\\
    \addlinespace[0.5em]
                 & tini \textbf{buddhadebake} parameshbarera abatAra balena
                 & POS: Noun; Suffixes: -ke\\
    \midrule
    \addlinespace[0.5em]
    \textbf{k2m}
                 & Ami mAke \textbf{chiThi} likhachhi
                 & POS: Noun\\
    \midrule
    \addlinespace[0.5em]
    \textbf{k2g}
                 & Ami \textbf{mAke} chiThi likhachhi
                 & POS: Noun; Suffixes: -ke\\
    \midrule
    \addlinespace[0.5em]
    \textbf{k3}
                 & gAdhAke \textbf{chAbuka} mAro
                 & POS: Noun\\
    \addlinespace[0.5em]
                 & gAdhAke \textbf{chAbuka} diYe mAro
                 & POS: Noun; Postposition: diYe\\
    \midrule
    \addlinespace[0.5em]
    \textbf{k4}
                 & rAm mohanke mArlo
                 & POS: Noun; Suffixes: -ke\\
    \midrule
    \addlinespace[0.5em]
    \textbf{k5}
                 & \textbf{nadIra} ghATa theke ghaTa bhese ela
                 & POS: Noun; Postposition: theke\\
    \addlinespace[0.5em]
                 & AmAra \textbf{ghara} theke mandirera chU.DA dekhA yAYa.
                 & POS: Noun; Postposition: theke\\
    \midrule
    \addlinespace[0.5em]
    \textbf{k7p}
                 & \textbf{bADite} phulera gAchha Achhe
                 & POS: Noun; Suffixes: -te\\
    \addlinespace[0.5em]
                 & \textbf{kArkhAnAy} kAj hoChe
                 & POS: Noun; Suffixes: -Ay\\
    \midrule
    \addlinespace[0.5em]
    \textbf{k7t}
                 & \textbf{bhore} sUrya oThe.
                 & POS: Noun; Suffixes: -e\\
    \addlinespace[0.5em]
                 & \textbf{ChotobelAy} luDo kHeltAm
                 & POS: Noun; Suffixes: -Ay\\
    \midrule
    \addlinespace[0.5em]
    \textbf{k7}
                 & tArA \textbf{sAhitye} paNDita
                 & POS: Noun; Suffixes: -ye\\
    \addlinespace[0.5em]
                 & tArA khuba \textbf{sukhe} Achhe
                 & POS: Noun; Suffixes: -e\\
    \midrule
    \addlinespace[0.5em]
    \textbf{rh}
                 & \textbf{bhaYe} bhule yAYa debatAra nAma.
                 & POS: Noun; Suffixes: -e\\
    \midrule
    \addlinespace[0.5em]
    \textbf{ru}
                 & ratana \textbf{unnatira} janya kaThora parishrama kare
                 & POS: Noun, Suffixes: -ra, Postposition: janya\\
    \midrule
    \addlinespace[0.5em]
    \textbf{r6v}
                 & \textbf{rAmera} ekaTA meYe Achhe
                 & POS: Noun; Suffixes: -era (genitive marker); Verb: Achhe\\
    \midrule
    \addlinespace[0.5em]
    \textbf{des}
                 & rAma bA.Di \textbf{giYechhila}
                 & POS: Verb; Suffixes: -echhila (nominative marker)\\
    \midrule
    \addlinespace[0.5em]
    \textbf{ras}
                 & Filler text
                 & POS: Noun; Suffixes: -e\\
    \addlinespace[0.5em]
                 & Filler text
                 & POS: Noun; Suffixes: -e\\
    \midrule
    \addlinespace[0.5em]
    \textbf{k*u}
                 & Filler text
                 & POS: Noun; Suffixes: -e\\
    \addlinespace[0.5em]
                 & Filler text
                 & POS: Noun; Suffixes: -e\\
    \midrule
    \addlinespace[0.5em]
    \textbf{k*s}
                 & Filler text
                 & POS: Noun; Suffixes: -e\\
    \addlinespace[0.5em]
                 & Filler text
                 & POS: Noun; Suffixes: -e\\
    \midrule
    \addlinespace[0.5em]
    \textbf{relc}
                 & Filler text
                 & POS: Noun; Suffixes: -e\\
    \addlinespace[0.5em]
                 & Filler text
                 & POS: Noun; Suffixes: -e\\
    \midrule
    \addlinespace[0.5em]
    \textbf{adv}
                 & Filler text
                 & POS: Noun; Suffixes: -e\\
    \addlinespace[0.5em]
                 & Filler text
                 & POS: Noun; Suffixes: -e\\
    \midrule
    \addlinespace[0.5em]
    \textbf{adj}
                 & Filler text
                 & POS: Noun; Suffixes: -e\\
    \addlinespace[0.5em]
                 & Filler text
                 & POS: Noun; Suffixes: -e\\
    \midrule
\caption{Discussion on AnnCorra tags}
\label{tab:anncorratable}
\end{longtable}

\section{Linguistic Challenges with Annotation}

\subproblem{a} Differential Object marking
\subproblem{b} Non-Nominative Subjects
\subproblem{c} Complex Predicates
\subproblem{d} Non-finite clauses: Conditional, Concessive, Relative, participial clauses
\subproblem{e} Ambiguity (Coordination, Attachment)
\subproblem{f} Ellipsis
\subproblem{g} Non-projectivity
	Ex 1: I saw a man yesterday who was singing
	Ex 2: A hearing is scheduled on the issue today
	Ex 3: To his wife, John gave a fantastic gift
	Ex 4: Which house, John bought?
\subproblem{h} Particles

\section{Tag Statistics}

\subsection{Tag and Markers}

For each tag, indicate the markers used to identify that tag and the number of tokens identified by each marker. Example:

There are a total of $100$ k1 tags. k1 comes with the markers $0, -ne$. $0$ marker is responsible for $56$ of the cases and $-ne$ is responsible for the remaining $44$.

\subsection{Markers and Tag}

For each marker, indicate the types of tags given to it and the number of cases for each tag. Example:

There are a total of $100$ tokens with $-ne$ marker. These tokens are marked with $k1, k2$. $k1$ marker is responsible for $32$ of the cases and $k2$ is responsible for the remaining $68$.

\subsection{N-gram of tags}
Include statistics about the frequency of n-gram of tags. Take $n$ in the range [2,4].
\newline \newline \newline
\textbf{NOTE: You have to do the above 3 exercises for both UD and AnnCorra. This analysis has to be done on the training set given for both the types of tagging in Assignment 4.}

\section{Error Analysis for automatic tagging}
In this section, describe the errors in the output of your model trained in Assignment 4 on the test data. Give possible solutions, if any, to mitigate these errors in the future. 

\section{Discussion}

\subsection{Comparison of UD and AnnCorra}
\subsection{Need for new tags}
\subsubsection{Intra-Chunk}
propose any new intra-chunk tags you can think of with examples
\subsubsection{Inter-Chunk}
propose any new inter-chunk tags you can think of with examples.

\end{document} 
